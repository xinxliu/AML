\documentclass[a4paper,UTF8]{article}
\usepackage{ctex}
\usepackage[margin=1.25in]{geometry}
\usepackage{color}
\usepackage{graphicx}
\usepackage{amssymb}
\usepackage{amsmath}
\usepackage{amsthm}
\usepackage{bm}
\usepackage{algorithm}  
\usepackage{algorithmic}  
%\usepackage[thmmarks, amsmath, thref]{ntheorem}
\theoremstyle{definition}
\newtheorem*{solution}{Solution}
\newtheorem*{prove}{Proof}
\usepackage{multirow}
\usepackage{url}
\usepackage{enumerate}
\renewcommand\refname{参考文献}

%--

%--
\begin{document}
\title{实验1. 度量学习实验报告}
\author{MF1733037,刘鑫鑫,\url{liuxx@lamda.nju.edu.cn}}
\maketitle

\section*{综述}
度量学习用于学习一个度量样本间距离的距离函数。以马氏距离为例:
\begin{equation}\label{mah}
dist^2_{mah}(\bm x_i,\bm x_j)=(\bm x_i-\bm x_j)^T\bm M(\bm x_i-\bm x_j)
\end{equation}
其中$\bm x_i,\bm x_j$是两个样本,$dist_{mah}(\bm x_i,\bm x_j)$则是这两个样本间的距离。对马氏距离的度量学习实际上就是学习矩阵$\bm M$。对$\bm M$的学习往往可以将其嵌入到某些分类器中。如近邻成分分析(Neighbourhood Component Analysis,NCA)就是将其嵌入到近邻分类器中。

\section*{任务1}
	\subsection*{度量函数学习目标}
		在任务1中使用近邻成分分析法。其优化目标为:
\begin{equation}
P = \mathop{\arg\min}_P 1-\sum^m_{i=1}\sum_{j\in\Omega_i}\frac{exp(-dist^2_{mah}(\bm x_i,\bm x_j))}{\sum_l exp(-dist^2_{mah}(\bm x_i,\bm x_l))}
\end{equation}
其中$\Omega_i$表示和样本$\bm x_i$标签一致的样本集合。考虑到矩阵$\bm M$是半正定对称矩阵,即存在矩阵$\bm P$使得$\bm M=\bm P^T\bm P$,于是优化目标可以写为:
\begin{equation}\label{opt-obj}
P = \mathop{\arg\min}_P 1-\sum^m_{i=1}\sum_{j\in\Omega_i}\frac{exp(-\Vert\bm P^T\bm x_i-\bm P^T\bm x_j\Vert_2^2)}{\sum_l exp(-\Vert\bm P^T\bm x_i-\bm P^T\bm x_l\Vert_2^2)}
\end{equation}
	\subsection*{优化算法}
对于式(\ref{opt-obj})所示的优化目标,可以采用随机梯度下降法。\\

设数据集为$D=\{(\bm x_1,y_1),(\bm x_2,y_2),\ldots,(\bm x_n,y_n)\}$。即对每个样本$\bm x_i$,设其标签为$y_i$。那么对于样本$\bm x_i$,我们要去最小化如下函数:
\begin{equation}
f(\bm x_i) = - \sum_{j\in\Omega_i}\frac{exp(-\Vert\bm P^T\bm x_i-\bm P^T\bm x_j\Vert_2^2)}{\sum_l exp(-\Vert\bm P^T\bm x_i-\bm P^T\bm x_l\Vert_2^2)}
\end{equation}
该函数是可导的,求出其对参数矩阵$\bm P$的导函数为:
\begin{equation}
\frac{\partial f}{\partial\bm P}=-2\bm P(p_i\sum_k p_{ik}\bm x_{ik}\bm x_{ik}^T-\sum_{j\in\Omega_i}p_{ij}\bm x_{ij}\bm x_{ij}^T)
\end{equation}
其中:
\[
p_{ij}=\frac{exp(-\Vert\bm P^T\bm x_i-\bm P^T\bm x_j\Vert_2^2)}{\sum_l exp(-\Vert\bm P^T\bm x_i-\bm P^T\bm x_l\Vert_2^2)}
\]
\[
p_i=\sum_{j\in\Omega_i}p_{ij}
\]
\[
\bm x_{ij} = \bm x_i-\bm x_j
\]
接下来,就可以应用随机梯度下降法来优化NCA的目标式(\ref{opt-obj})了:
\begin{algorithm}
\caption{SGD for NCA}
\textbf{Input:}\\ 
DataSet $D=\{(\bm x_1,y_1),(\bm x_2,y_2),\ldots,(\bm x_n,y_n)\}$\\
learning rate for SGD: lr\\
maximum iteration for SGD:max\_iter\\
\textbf{Output:} Matrix $\bm P$

\begin{algorithmic}[1]
\STATE initialize $\bm P$
\STATE initialize max\_iter
\FOR{$iter \in (1:max\_iter)$}
\STATE i = iter\% n
\STATE calculate $\frac{\partial f}{\partial\bm P}$
\STATE update $\bm P$: $P\leftarrow P-lr*\frac{\partial f}{\partial\bm P}$
\ENDFOR
\end{algorithmic}
\end{algorithm}

具体代码见附件中myDML.py。其中myDML.train()函数用于训练样本,myDML.distance()函数用于计算样本间距离。使用时注意要先训练样本再调用距离函数,否则会出现错误。主要的矩阵运算均调用numpy包里的相关函数。

对于任务1数据集,笔者推荐的超参数为$lr = 0.01,max\_iter=20000$。

\section*{任务2}
在该任务中,笔者仍然使用任务1中的NCA方法去训练样本并计算距离,优化方法也是SGD。在设定合适的超参数后,该方法错误率低于欧几里得距离的错误率约$6\% - 8\%$。具体测试结果见附件evaluation.txt

对于任务2数据集,笔者推荐的超参数为$lr=0.005,max\_iter=182000$








































\end{document}